\documentclass[a4paper,14pt]{extreport}
\usepackage[utf8]{inputenc}
\usepackage[T1, T2A]{fontenc}
\usepackage{csquotes}
\usepackage[english, russian]{babel}
\usepackage{filecontents}
\usepackage{amsmath}
% \usepackage{mathtools}

% NOTE: Check for � always! They seem to creep up.

% Bibliography style
\usepackage[
style=gost-numeric, %or just numeric
backend=biber,
%sorting=ynt,
language=auto
]{biblatex}

\begin{filecontents}{fake_bib.bib}
  % bibliopraphy does here
  
	@online{
		oecdCycleExtraction,
		author = "Nilsson, R. and Gyomai, G.",
		title = "Cycle Extraction. A comparison of the Phase-Average Trend method, the Hodrick–Prescott and Christiano–Fitzgerald filters",
		publisher = "OECD Statistics Working Paper [Electronic resource]",
		year = "2011",
		url = "http://dx.doi.org/10.1787/5kg9srt7f8g0-en",
		urldate = "2016-05-27"
	}
	
	@article{
		nberDevelopment,
		author = "Moore, G. and Zarnowitz, V. ",
		title = "The Development and Role of the National Bureau of Economic Research's Business Cycle Chronologies",
		journal = "The American Business Cycle: Continuity and Change; National Bureau of Economic Research",
		pages = "735-780",
		year = "1986"
	}

	@article{
		hamNewApproach,
		author = "Hamilton, J.D.",
		title = "A New Approach to the Economic Analysis of Nonstationary Time Series and the Business Cycle",
		journal = "Econometrica",
		volume = "57",
		number = "2",
		pages = "357-384",
		year = "1989"
	}

	@report{
		esiMaking,
		author = "",
		aauthor = "Малюгин, В.И.",
		title = "Разработка системы опережающих экономических индикаторов и экономических диффузных индексов для основных видов экономической деятельности и экономики Республики Беларусь в целом с использованием экономико-математических, эконометрических методов и моделей на основе данных системы мониторинга предприятий Национального банка Республики Беларусь: ",
		type = " отчет о НИР (заключ.)",
		institution = "НИИ ППМИ; рук. В.И. Малюгин",
		date = "2017",
		number = "ГР 20162817",
		location = "Минск",
		pagetotal = "142",
	}
		
	@inproceedings{
		esiExtra,
		author = "Малюгин, В.И. and others",
		title = "Модельные и инструментальные средства для построения и применения индекса экономических настроений белорусской экономики",
		booktitle = "Проблемы прогнозирования и государственного регулирования социально-экономического развития: материалы XVII Международной конференции",
		date = "2017",
		volume = "1",
		pages = "178-188",
		organization = "НИЭИ Минэкономики Республики Беларусь",
		OPTeventtitle = "Проблемы прогнозирования и государственного регулирования социально-экономического развития"
	}

	@article{
		malNovopMSVARX,
		author = "Malugin, V. and Novopoltsev, A.",
		title = "Statistical Estimation and Classification Algorithms for Regime-Switching VAR Model with Exogenous Variables",
		journal = "Austrian Journal of Statistics",
		volume = "46",
		pages = "47-56",
		year = "2017"
	}

	@article{
		malVARforCycles,
		author = "Малюгин, В.И.",
		title = "Об использовании векторных авторегрессионных моделей с переключающимися состояниями для анализа и прогнозирования циклов экономической активности",
		journal = "Экономика. Моделирование. Прогнозирование",
		volume = "9",
		pages = "183-196",
		year = "2015"
	}

	@article{
		malNovopHiddenMarkov,
		author = "Малюгин, В.И. and Новопольцев, А.Ю.",
		title = "Анализ многомерных статистических моделей с неоднородной структурой в случае скрытой марковской зависимости состояний",
		journal = "Известия НАН Беларуси",
		volume = "1",
		number = "2",
		pages = "26-36",
		year = "2015"
	}

	@online{
		statsmodels,
		title = "Python package Statsmodels on statsmodels.org",
		author = "Perktold, J. and Seabold, S. and statsmodels-developers", 
		year = "2017",
		url = "http://www.statsmodels.org/stable/index.html"
	}
	@online{
		fulton_statespace,
		title = "State space modeling in Python",
		author = "Fulton, C.",
		year = "2016",
		url = "http://www.chadfulton.com/topics/state_space_python.html"
	}

\end{filecontents}

\addbibresource{fake_bib.bib}

% Оформление глав, разделов и т.д.
\makeatletter

% Не подавлять абзацный отступ в главах
\renewcommand{\chapter} {
  \cleardoublepage\thispagestyle{plain}
  \global\@topnum=0 \@afterindenttrue \secdef\@chapter\@schapter
}

% Оформление нумерованных глав
\renewcommand{\@makechapterhead}[1] {
  \vspace{36pt} % Пустое место вверху страницы
  {
    \centering
    \parindent=18pt
    \normalfont\Large\bfseries
    \chaptername ~ \thechapter{} \par % Номер главы
    #1 \par % Заголовок текста с новой строки
    \nopagebreak % Не отрываем заголовок от текста
    \vspace{36pt} % Пустое место между заголовком и текстом
  }
}

% Оформление ненумерованных глав
\renewcommand{\@makeschapterhead}[1] {
  \vspace{36pt} % Пустое место вверху страницы
  {
    \centering
    \parindent=18pt
    \normalfont\Large\bfseries #1 \par
    \nopagebreak % чтобы не оторвать заголовок от текста
    \vspace{25pt} % между заголовком и текстом
  }
}

% Оформление разделов
\renewcommand{\section} {
    \@startsection{section}
                  {1}
                  {18pt}
                  {3.5ex plus 1ex minus .2ex}
                  {2.3ex plus .2ex}
                  {\normalfont\Large\bfseries\raggedright}
}

% Оформление подразделов
\renewcommand{\subsection} {
  \@startsection{subsection}
                {2}
                {18pt}
                {3.25ex plus 1ex minus .2ex}
                {1.5ex plus .2ex}
                {\normalfont\large\bfseries\raggedright}
}

% Оформление подподразделов
\renewcommand{\subsubsection} {
  \@startsection{subsubsection}
                {3}
                {18pt}
                {3.25ex plus 1ex minus .2ex}
                {1.5ex plus .2ex}
                {\normalfont\large\bfseries\raggedright}
}


\addto\captionsrussian{\renewcommand\figurename{Рисунок}}

%Оформление подписи рисунка
%\renewcommand \thefigure{\thesection.\@arabic\c@figure }
%чтобы номер рисунка содержал номер главы (например, рисунок 2.1) надо закомментировать предыдущую строку
\renewenvironment{figure}{%
\let\@makecaption\@makefigurecaption
\@float{figure}
}%
{%
\addtocontents{lof}{ {\vskip 0.2em} }
\end@float
}

\newcommand{\@makefigurecaption}[2]{%
\vspace{\abovecaptionskip}%
\sbox{\@tempboxa}{\normalsize #1 --- \normalsize #2}%
\ifdim \wd\@tempboxa >\hsize {\center\hyphenpenalty=10000\normalsize #1 --- \normalsize #2 \par}%
\else \global\@minipagefalse \hbox to \hsize
{\hfil \hyphenpenalty=10000 \normalsize #1 --- \normalsize #2\hfil}%
\fi \vspace{\belowcaptionskip}}


%Оформление подписи таблицы
%\renewcommand{\thetable}{\@arabic\c@table}
%чтобы номер таблицы содержал номер главы (например, таблица 2.1) надо закомментировать предыдущую строку
\renewenvironment{table}{%
\let\@makecaption\@maketablecaption
\@float{table}}%
{%
\addtocontents{lot}{ {\vskip 0.4em} }%
\end@float%
}
%

\newlength\abovetablecaptionskip
\newlength\belowtablecaptionskip
\newlength\tableparindent
\setlength\abovetablecaptionskip{10\p@}
\setlength\belowtablecaptionskip{0\p@}
\setlength\tableparindent{18\p@}
\newcommand{\@maketablecaption}[2]{
  \vskip\abovetablecaptionskip
  \hskip\tableparindent \normalsize #1~---\ \normalsize #2\par
  \vskip\belowtablecaptionskip
}


\newcommand{\fakechapter}[1] {
	\chapter*{#1}
	\addcontentsline{toc}{chapter}{#1}
	\markboth{#1}{#1}
}

% Parameters

\newcommand*{\authorfirst}[1]{\gdef\@authorfirst{#1}}
\newcommand*{\@authorfirst}{}

\newcommand*{\authorlast}[1]{\gdef\@authorlast{#1}}
\newcommand*{\@authorlast}{}

\newcommand*{\mentor}[1]{\gdef\@mentor{#1}}
\newcommand*{\@mentor}{}

\newcommand*{\mentorjob}[1]{\gdef\@mentorjob{#1}}
\newcommand*{\@mentorjob}{}

\newcommand*{\faculty}[1]{\gdef\@faculty{#1}}
\newcommand*{\@faculty}{}

\newcommand*{\subfaculty}[1]{\gdef\@subfaculty{#1}}
\newcommand*{\@subfaculty}{}

\newcommand*{\specialty}[1]{\gdef\@specialty{#1}}
\newcommand*{\@specialty}{}

\newcommand*{\reviewer}[1]{\gdef\@reviewer{#1}}
\newcommand*{\@reviewer}{}

% Values

\title{Разработка алгоритмов машинного обучения для построения моделей с переключением состояний}
\authorlast{Макаревич}
\authorfirst{Анатолий Сергеевич}
\author{\@authorlast \@authorfirst}

\mentor{Малюгин Владимир Ильич}
\mentorjob{доцент, кандидат физико-математических наук, кафедра ММАД}
\faculty{Факультет Прикладной Математики и Информатики}
\subfaculty{Кафедра Математического Моделирования и Анализа Данных}
\specialty{ПКАД (Прикладной Компьютерный Анализ Данных)}


\begin{document}

% Create title page

\begin{titlepage}
	\begin{center}
		\small{МИНИСТЕРСТВО ОБРАЗОВАНИЯ РЕСПУБЛИКИ БЕЛАРУСЬ}\\
		\small{БЕЛОРУССКИЙ ГОСУДАРСТВЕННЫЙ УНИВЕРСИТЕТ}\\
		\small{\MakeUppercase{\@faculty}}\\
		\@subfaculty
	\end{center}
																																																																																								  
	\vspace{5em}
																																																																																								  
	\begin{center}
		\MakeUppercase{\@authorlast} \@authorfirst \\
		\vspace{1em}
		\textbf{\MakeUppercase{\@title}} \\
		\vspace{2em}
		Магистерская диссертация \\
		специальнось \@specialty
	\end{center}
																																																																																								
	\vspace{2em}
	\begin{flushright}
		\begin{minipage}[H]{0.4\textwidth}
			\begin{flushleft}
				Научный руководитель \\
				\@mentor \\
				\@mentorjob
			\end{flushleft}
		\end{minipage}
	\end{flushright}
																																																																													
	\vspace{3em}
																																																																													
	\vfill
																																																																															
	\begin{flushleft}
		\begin{minipage}[H]{0.5\textwidth}
			\begin{flushleft}
				Допущено к защите \\
				<<\rule{1cm}{1pt}>> \rule{4cm}{1pt} \the\year г. \\
				Зав. Кафедрой \\
				\rule{6cm}{1pt} \\
				\rule{6cm}{1pt}
			\end{flushleft}
		\end{minipage}
	\end{flushleft}
																																																																																								
	\vspace{1em}
																																																																																								  
	\begin{center}
		Минск, \the\year
	\end{center}
\end{titlepage}


% ABSTRACT

% \begin{abstract}
% 	Аннотация на русском.
% \end{abstract}

% \selectlanguage{english}
    
% \begin{abstract}
% 	Abstract in English.
% \end{abstract}
    
\selectlanguage{russian}

% Общая характеристика работы
% «Общая характеристика работы» содержит:
% перечень ключевых слов; 
% цель, задачи, объект и предмет исследования;
% формулировку полученных результатов и их новизну;
% сведения о структуре магистерской диссертации. 
% Перечень ключевых слов характеризует основное содержание магистерской диссертации и включает 10-15 слов в именител/ьном падеже, написанных через запятую в строку прописными буквами. 

\fakechapter{Реферат}

Работа: N стр., N табл., N рис., N источников, N приложений

\MakeUppercase{временные ряды, модели с переключением состояний, анализ поворотных точек экономических циклов}

Объекты исследования -- математические модели временных рядов, использующие переключение состояния; индикаторы экономики Республики Беларусь.

Цель работы -- исследование свойств моделей с переключением состояния (а также подклассы MS-VARX, IS-VARX) и их применение к задачам эконометрического моделирования на примере определения поворотных точек бизнес-цикла ВВП Республики Беларусь.

Методы исследования -- компьютерное моделирование временных рядов как сгенерированных, так и индикаторов экономики Республики Беларусь.

В результате исследования получены новые модели для темпов роста ВВП РБ, подтверждение опережающего характера индикатора экономических настроений (ИЭН), а также новые подходы к оценке параметров более общих RS-моделей.


% ОГЛАВЛЕНИЕ

\clearpage
% Название содержания меняем
\renewcommand{\contentsname}{Содержание}
\tableofcontents

% Перечень условных обозначений, символов и терминов
\fakechapter{Условные обозначения и термины}

\section{Термины и сокращения}

\textbf{AR, VAR, VARX} -- авторегрессионные модели, соответственно: одномерная, векторная, векторная с экзогенными переменными.

\textbf{IS-} -- модель с независимым переключением состояний (от англ. independent switching).

\textbf{MS-} -- модель с Марковским переключением состояний (от англ. Markov switching).

\textbf{GDP, ESI} -- макроэкономические индикаторы, ВВП и ИЭН (индекс экономических настроений) соответственно.

\section{Обозначения в формулах}

$y$ -- эндогенная (моделируемая) переменная, случайная величина, возможно векторная.

$x$ -- экзогенная переменная, возможно векторная.

$y_t, \: x_t$ -- реализация переменных в момент времени $t$.

$\hat{y}_t$ -- прогноз эндогенной переменной на момент времени $t$.

$l_t = l(t) \in \overline{1,L}$ -- латентная (ненаблюдаемая) переменная состояния.

$L$ -- количество классов состояния (режимов).


% Введение
% Во «Введении» (объем до 3 страниц) обосновывается актуальность темы, ее значение, выбор направления исследования, показывается необходимость проведения исследований по данной теме для решения конкретной проблемы (задачи), развития конкретных направлений в соответствующих областях науки, отраслях экономики
\fakechapter{Введение}

В работе представляется многостороннее исследование моделей временных рядов с переключением состояний (regime-switching models, RS-модели), в частности авторегрессионные модели (семейства RS-VARX) с независимым (IS-VARX) или Марковским (MS-VARX) переключением состояний. 

TODO: Текст об экономических циклах и применении RS-моделей 

TODO: Текст о вероятностном программировании и возможностях оценки RS-моделей ими

\dots


% Основная часть
\chapter{Обзорная часть (TODO: Rename)}
% TODO: Rename

\section{Модели с переключением}

Модели со скрытыми переменными (latent variables) по определению включают в себя структурное предположение: в моделируемом процессе существует хотя бы одна переменная, которая не наблюдается или не может наблюдаться, но которая влияет на наблюдаемые/измеримые переменные. Такие модели используются в экономике, машинном обучении (особенно при обработке естественного языка), биологии, психологии и т. д.

В данной работе рассматриваются модели временных рядов, где скрытая переменная является дискретной и категориальной с возможными значениями $l \in \overline{1,L} = \{1,2,\dots,L\}$. Интерпретация этой переменной -- класс состояния (regime), который определяет подмодель, по которой определяются другие переменные. Чаще всего эти подмодели имеют одинаковую структуру, но параметры оцениваются отдельно для каждого класса. В таких случаях модели с переключением состояний (regime switching models / RS-модели) – подкласс моделей со структурными изменениями.

\section{Экономические циклы и поворотные точки}

В рамках концепции экономического цикла или <<бизнес-цикла>> (business cycle), используемой в НБЭИ (Национальное бюро экономических исследований) США \cite{nberDevelopment}, подразумевается последовательная смена двух фаз базового экономического индикатора, называемых периодами <<роста>> (growth) и <<спада>> (recession) экономической активности \cite{oecdCycleExtraction}. Другие популярные определения экономических циклов могут состоять из трех или четырех фаз.

Моменты смены фазы роста на фазу спада, и наоборот, называются <<поворотными точками>> бизнес-цикла. Одной из ключевых задач анализа и прогнозирования экономической активности является разработка систем раннего обнаружения смены фаз экономических циклов.  Ранние работы (CITE) рассматривали различные подходы оценивания поворотных точек для экономики Республики Беларусь. В качестве базового индикатора рассматривается реальное ВВП (GDP) Республики Беларусь.

В ходе НИР \cite{esiMaking} был построен композитный опережающий индикатор ESI (Economic Sentiment Index / Индекс Экономических Настроений / ИЭН). Для выделения циклических компонент ESI и GDP в ней был доработан метод, основанный на двойном применении фильтра Ходрика—Прескотта. Из циклической компоненты ряда уже нетрудно добыть поворотные точки \cite{esiMaking, esiExtra}. 

У этого подхода есть ряд недостатков (CITE), самым главным который является невозможность прогнозирования рядов в будущее (вследствие двусторонности фильтра). Автором была приведена альтернативная методика, основанная на использовании фильтра Хамильтона (CITE), которая исправляет часть недостатков.

Существует и альтернативный подход оценивания поворотных точек, основанный на моделях с переключением состояний \cite{hamNewApproach}. В частности, используются авторегрессионные модели с переключением состояния и экзогенными переменными (класс RS-VARX) для которых разработаны алгоритмы оценивания параметров и классов состояния \cite{malNovopMSVARX}. Каждое отдельное состояние $l$ интерпретируется как часть экономического цикла; например, при $L=2$, эти два класса интерпретируются как фазы роста и спада. 

Далее в работе рассмотрены подробнее как сам класс моделей RS-VARX, так и применение к задаче оценивания поворотных точек экономики Республики Беларусь. 


\chapter{Эконометрические модели с переключением состояний и их применение в задачах анализа бизнес-цикла}

\section{Общая характеристика моделей с переключением состояния}

Модели с переключением состояния (далее – RS-модели) используются для исследования и прогнозирования временных рядов. Вектор эндогенных (моделируемых) переменных обозначим $y$, вектор экзогенных -- $x$, скрытую переменную класса состояния -- $l$.
Самая общая формулировка моделей RS-моделей для временных рядов следующая:
\[
	y_t \sim F_{l(t)}(t, y_{t-1}, \dots, y_0, x_t, x_{t-1}, \dots, x_0) 
\]
\[
	l(t) = l_t \sim S(t, l_{t-1}, \dots, l_0, y_{t-1}, \dots, y_0, x_t, \dots, x_0) 
\]

где $l(t) = l_t$ -- класс состояния в момент времени $t$, $S(\cdot)$ -- функция изменения состояния, а $\{F_l(\cdot)\}$ -- функции условного распределения $y$ при значении $l \in \overline{1,L}$. 

RS-модели, таким образом, включают очень широкий класс возможных моделей, однако на практике используются конкретные классы моделей. Особое внимание отделим авторегрессионным моделям RS-VARX, которые описываются следующей функцией условного распределения:
\[
	y_{t}=c_{l(t)} + \sum_{i=1}^{p} A_{i,l(t)} y_{t-i} + B_{l(t)} x_{t} + \eta_{t, l(t)} 
\]

где 
$y_t$ -- эндогенный вектор размерности $n$, 
$x_{t}$ -- экзогенная векторная переменная размерности $m$,
$p$ -- порядок авторегрессии, 
$c_{l}$ -- вектор констант,
$A_{i,l}$ ($n \times n$ матрицы) и $B_{l}$ ($n \times m$ матрицы) -- матрицы авторегрессии и регрессии соответственно,
и $\eta_{t, l} \sim N_n(0, \Sigma_{l}) $ -- нормально–распределенный белый шум.

Все переменные, индексированные с $l$, могут зависеть от текущего состояния (<<переключаются>>); для уменьшения количества параметров, возможно убрать это переключение для некоторых из переменных.


На процесс состояния и $S(\cdot)$ также добавляются предположения. В работе рассматриваются модели с независимым переключением состояния IS-VARX:

\[ 
	l_t \sim \mathit{Cat}(L, p) 
\]

где $p$ -- вектор вероятности классов (размерности $L$), $\sum_{i=1}^{L}{p_i} = 1, p_i > 0$.

Также рассматриваются модели с Марковским переключением MS-VARX. 
В этом случаи $l_t$ описывается однородной цепью Маркова:

\[
	l_t \sim \mathit{Cat}(L, M_{l(t-1), \cdot})
\]

где $M$ -- матрица вероятности перехода классов (размерности $L \times L$) с условиями:

\[
	M=
	\left[ {\begin{array}{cccc}
			m_{1,1} & m_{1,2} & ... & m_{1,L} \\
			m_{2,1} & m_{2,2} & ... & m_{2,L} \\
			... & ... & ... & ... \\
			m_{L,1} & m_{L,2} & ... & m_{L,L} \\
		\end{array} } \right]
	, \quad
	\sum_{i=1}^{L} m_{i,j} = 1 \quad \forall j \in \overline{1,L}
	,
\]
\[
	0 \le m_{i,j} \le 1 \quad \forall i, j \in \overline{1,L},
\]

а $M_{i, \cdot}$ -- $i$-я строка матрицы. При $L=2$ можно упростить ее структуру:

\[
	M=
	\left[ {\begin{array}{cc}
			\sigma_{1} & 1-\sigma_{2} \\
			1-\sigma_{1} & \sigma_{2} \\
		\end{array} } \right]
	, \quad 
	0 \le \sigma_{1} \le 1
	, \quad 
	0 \le \sigma_{2} \le 1
\]

Эти модельные предположения структур процессов для $y_t$, $l_t$ позволяют использовать определенные алгоритмы оценки параметров, которые описаны далее.


\section{Оценка параметров RS-VARX моделей с размеченной выборкой}

TODO: Рассказать про дискриминантный анализ

\section{EM-алгоритмы для IS-VARX и MS-VARX}

TODO: Вставить теорию и пример EM-алгоритма

\section{Применение MS-VARX к задаче оценки поворотных точек бизнес-цикла экономики Республики Беларусь}

\subsection{Описание исходных данных}

Как было описано в первой части данной работы, RS-модели могут применяться к задаче определения фаз бизнес-цикла. В качестве базового индикатора бизнес-цикла рассматривается реальный ВВП (GDP) Республики Беларусь в ценах 2014 г. в месячном исчислении; он же рассматривается в качестве моделируемой переменной. Вышеописанный Индекс Экономических Настроений (ESI) используется в качестве экзогенной переменной. Так как ESI является опережающим индикатором (поворотные точки ESI опережают поворотных точек GDP на 2-4 месяца (CITE)), рассматриваются варианты моделей со включением различных лагов этой переменной (опережение рассматривалось от 0 до 5 периодов).

\subsection{Исследование рядов GGDP, GESI статистическими тестами}

Для тестирования интегрированности рядов GGDP и GESI (и их первых разностей) использовались тесты, допускающие наличие структурных изменений. Используя тест BPUR (Breakpoint Unit Root), гипотеза об интегрированности с <<инновационными аномалиями>> (сдвигами) не была отклонена. С помощью теста Баи – Перрона (Bai – Perron tests for sequentially determined breaks) и построении моделей со структурными изменениями (Breakpoint Least Squares) выявлен феномен <<кобрейкинга>> (одновременного структурного изменения) рядов GGDP и GESI. Более подробно процедура описана в (CITE), где также построены две ARX модели (формулы 4 и 5 в статье), не включающие изменения в параметрах т. к.  структурные изменения в GGDP и GESI совпадают. Также было проведено тестирование на коинтегрированность этих рядов, в котором гипотеза о коинтеграции не отклонилась (CITE). 

Принятая методология определения фаз бизнес-цикла экономики Республики Беларусь предполагает две фазы (<<спад>> и <<подъем>>), поэтому фиксируется количество классов $L=2$ которые интерпретируются соответствующим образом. В данной работе везде использовался класс моделей MS-ARX вследствие предыдущих работ автора и руководителя по данной тематике (CITE).

Цель экспериментальной части (CITE) состоит в оценивании предиктивных способностей моделей MS-ARX, включающие вышеописанный опережающий индикатор. Ряды GDP, ESI содержат сезонные эффекты. В представляемых моделях использовались годовые темпы ростов этих рядов GGDP и GESI соответственно, которые можно рассматривать как сезонно скорректированные ряды. Их циклические компоненты демонстрируют опережающий характер GESI\_C по отношению к GGDP\_C (CITE, PICTURE).

\subsection{Модели MS-ARX}


При моделировании GGDP рассматривались некоторые модели семейства MS-ARX, включающие различные структурные предположения и экзогенные переменные. Везде рассматривалось $L=2$ класса и порядок авторегрессии не более $p=2$, т. е. класс MS(2)-ARX(2). В дальнейших формулах используются обозначения $y_t = \text{GGDP}_t$, $x_t = \text{GESI}_t$, и $\eta_t$ -- инновационный процесс.

Из рассмотренных моделей, следующие сошлись и дали лучшие результаты:

\[
	M.0. \quad y_t = c_{l(t)} + \alpha_{l(t), 1} y_{t-1} + \alpha_{l(t), 2} y_{t-2} + \eta_t
\]

Модель $M.0$ не содержит экзогенных факторов и предполагает, что циклические изменения обусловлены аномалиями в инновационном процессе $\eta$ и ведут к изменениям среднего уровня GGDP.

Для моделей $M.1$ - $M.3$ отключено изменение в авторегрессионных коэффициентах $\alpha_i$; вследствие незначимости этих коэффициентов, они исключены в итоговых вариантах моделей.
% TODO: Проверить правильно ли это – может просто не включали изначально?

\[
	M.1. \quad y_t = c_{l(t)} + \alpha_1 y_{t-1} + \alpha_2 y_{t-2} + \beta_{l(t), 1} t + \eta_t
\]

Модель $M.1$ допускает циклические изменения в трендовой компоненте. 


В основе $M.2$ лежит долгосрочная коинтеграционная зависимость между GGDP и GESI, включающая линейный тренд:

\[
	M.2. \quad y_t = c_{l(t)} + \alpha_1 y_{t-1} + \alpha_2 y_{t-2} + \beta_{l(t), 1} t + \beta_{l(t), 2} x_{t} + \eta_t
\]

$M.3$ включает такую же структуру, однако включает опережающую экзогенную переменную GESI(-4):

\[
	M.3. \quad y_t = c_{l(t)} + \alpha_1 y_{t-1} + \alpha_2 y_{t-2} + \beta_{l(t), 1} t + \beta_{l(t), 2} x_{t-4} + \eta_t
\]

Все коэффициенты при экзогенных переменных зависят от класса состояния $l_t$.

В таблицах [TABLE] и [TABLE] приведены характеристики оцененных моделей, включая коэффициенты, интерпретацию фаз <<роста>> и <<падения>>, и вероятности переходов. На основании этих данных можно сделать следующие выводы:

\begin{enumerate}
	\item Свободный член и включенные в модели экзогенные переменные действительно подвержены циклическим изменениям во всех моделях.
	\item В моделях $M.0$, $M.1$, $M.2$ в состоянии <<спад>> GGDP чувствительна к изменениям в свободном члене и всех включенных переменных (тренд, GESI, авторегрессионная часть), а в состоянии <<рост>> влияет только свободный член.
	\item В модели $M.3$ с опережающей переменной GESI(-4) все коэффициенты значимые, и сама модель обладает наилучшей предиктивной способности при определении двух классов состоянии.
\end{enumerate}


Графическое представление результатов экспериментов дано на рисунках [PICTURES]. На них представлены:

\begin{itemize}
	\item динамика годовых темпов роста реально ВВП (GGDP),
	\item интервалы соответствующие классам <<роста>> и <<спада>>,
	\item прогнозы моделей для периода оценивания,
	\item условные прогнозы моделей вне этого периода (для валидации).
\end{itemize}

% TODO: Рассказать о сглаженных и несглаженных вероятностях.


\chapter{Применение методов машинного обучения для построения моделей}


\section{Байесовские методы моделирования и оценок}

TODO: Описать Байесовские оценки параметров


\section{Вероятностное программирование (probabilistic programming)}

TODO: Define probabilistic programming


\section{Применение вероятностного программирования к RS-моделям}

TODO: Describe RS model as a probabilistic graphical model

TODO: Showcase experiment on simulated data


% Заключение (выводы)
\fakechapter{Заключение}

TODO: Conclusion.


% Список использованных источников
% \fakechapter{Список использованных источников}
\printbibliography[title=Список использованных источников]
\addcontentsline{toc}{chapter}{Список использованных источников}
\markboth{Список использованных источников}{Список использованных источников}

% Графический материал

% Программы (при необходимости)

% Приложения (при необходимости)


\end{document}