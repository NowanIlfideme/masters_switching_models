\documentclass[a4paper,14pt]{extreport}
\usepackage[utf8]{inputenc}
\usepackage[T1, T2A]{fontenc}
\usepackage{csquotes}
\usepackage[english, russian]{babel}
\usepackage{filecontents}

% Bibliography style
\usepackage[
style=gost-numeric, %or just numeric
backend=biber,
%sorting=ynt,
language=auto
]{biblatex}

\begin{filecontents}{fake_bib.bib}
  % bibliopraphy does here
\end{filecontents}

\addbibresource{fake_bib.bib}

% Оформление глав, разделов и т.д.
\makeatletter

% Не подавлять абзацный отступ в главах
\renewcommand{\chapter} {
  \cleardoublepage\thispagestyle{plain}
  \global\@topnum=0 \@afterindenttrue \secdef\@chapter\@schapter
}

% Оформление нумерованных глав
\renewcommand{\@makechapterhead}[1] {
  \vspace{36pt} % Пустое место вверху страницы
  {
    \centering
    \parindent=18pt
    \normalfont\Large\bfseries
    \chaptername ~ \thechapter{} \par % Номер главы
    #1 \par % Заголовок текста с новой строки
    \nopagebreak % Не отрываем заголовок от текста
    \vspace{36pt} % Пустое место между заголовком и текстом
  }
}

% Оформление ненумерованных глав
\renewcommand{\@makeschapterhead}[1] {
  \vspace{36pt} % Пустое место вверху страницы
  {
    \centering
    \parindent=18pt
    \normalfont\Large\bfseries #1 \par
    \nopagebreak % чтобы не оторвать заголовок от текста
    \vspace{25pt} % между заголовком и текстом
  }
}

% Оформление разделов
\renewcommand{\section} {
    \@startsection{section}
                  {1}
                  {18pt}
                  {3.5ex plus 1ex minus .2ex}
                  {2.3ex plus .2ex}
                  {\normalfont\Large\bfseries\raggedright}
}

% Оформление подразделов
\renewcommand{\subsection} {
  \@startsection{subsection}
                {2}
                {18pt}
                {3.25ex plus 1ex minus .2ex}
                {1.5ex plus .2ex}
                {\normalfont\large\bfseries\raggedright}
}

% Оформление подподразделов
\renewcommand{\subsubsection} {
  \@startsection{subsubsection}
                {3}
                {18pt}
                {3.25ex plus 1ex minus .2ex}
                {1.5ex plus .2ex}
                {\normalfont\large\bfseries\raggedright}
}


\addto\captionsrussian{\renewcommand\figurename{Рисунок}}

%Оформление подписи рисунка
%\renewcommand \thefigure{\thesection.\@arabic\c@figure }
%чтобы номер рисунка содержал номер главы (например, рисунок 2.1) надо закомментировать предыдущую строку
\renewenvironment{figure}{%
\let\@makecaption\@makefigurecaption
\@float{figure}
}%
{%
\addtocontents{lof}{ {\vskip 0.2em} }
\end@float
}

\newcommand{\@makefigurecaption}[2]{%
\vspace{\abovecaptionskip}%
\sbox{\@tempboxa}{\normalsize #1 --- \normalsize #2}%
\ifdim \wd\@tempboxa >\hsize {\center\hyphenpenalty=10000\normalsize #1 --- \normalsize #2 \par}%
\else \global\@minipagefalse \hbox to \hsize
{\hfil \hyphenpenalty=10000 \normalsize #1 --- \normalsize #2\hfil}%
\fi \vspace{\belowcaptionskip}}


%Оформление подписи таблицы
%\renewcommand{\thetable}{\@arabic\c@table}
%чтобы номер таблицы содержал номер главы (например, таблица 2.1) надо закомментировать предыдущую строку
\renewenvironment{table}{%
\let\@makecaption\@maketablecaption
\@float{table}}%
{%
\addtocontents{lot}{ {\vskip 0.4em} }%
\end@float%
}
%

\newlength\abovetablecaptionskip
\newlength\belowtablecaptionskip
\newlength\tableparindent
\setlength\abovetablecaptionskip{10\p@}
\setlength\belowtablecaptionskip{0\p@}
\setlength\tableparindent{18\p@}
\newcommand{\@maketablecaption}[2]{
  \vskip\abovetablecaptionskip
  \hskip\tableparindent \normalsize #1~---\ \normalsize #2\par
  \vskip\belowtablecaptionskip
}


\newcommand{\fakechapter}[1] {
	\chapter*{#1}
	\addcontentsline{toc}{chapter}{#1}
	\markboth{#1}{#1}
}

% Parameters

\newcommand*{\authorfirst}[1]{\gdef\@authorfirst{#1}}
\newcommand*{\@authorfirst}{}

\newcommand*{\authorlast}[1]{\gdef\@authorlast{#1}}
\newcommand*{\@authorlast}{}

\newcommand*{\mentor}[1]{\gdef\@mentor{#1}}
\newcommand*{\@mentor}{}

\newcommand*{\mentorjob}[1]{\gdef\@mentorjob{#1}}
\newcommand*{\@mentorjob}{}

\newcommand*{\faculty}[1]{\gdef\@faculty{#1}}
\newcommand*{\@faculty}{}

\newcommand*{\subfaculty}[1]{\gdef\@subfaculty{#1}}
\newcommand*{\@subfaculty}{}

\newcommand*{\specialty}[1]{\gdef\@specialty{#1}}
\newcommand*{\@specialty}{}

\newcommand*{\reviewer}[1]{\gdef\@reviewer{#1}}
\newcommand*{\@reviewer}{}

% Values

\title{Разработка алгоритмов машинного обучения для построения моделей с переключением состояний}
\authorlast{Макаревич}
\authorfirst{Анатолий Сергеевич}
\author{\@authorlast \@authorfirst}

\mentor{Малюгин Владимир Ильич}
\mentorjob{доцент, кандидат физико-математических наук, кафедра ММАД}
\faculty{Факультет Прикладной Математики и Информатики}
\subfaculty{Кафедра Математического Моделирования и Анализа Данных}
\specialty{ПКАД (Прикладной Компьютерный Анализ Данных)}


\begin{document}

% Create title page

\begin{titlepage}
	\begin{center}
		\small{МИНИСТЕРСТВО ОБРАЗОВАНИЯ РЕСПУБЛИКИ БЕЛАРУСЬ}\\
		\small{БЕЛОРУССКИЙ ГОСУДАРСТВЕННЫЙ УНИВЕРСИТЕТ}\\
		\small{\MakeUppercase{\@faculty}}\\
		\@subfaculty
	\end{center}
																																																														  
	\vspace{5em}
																																																														  
	\begin{center}
		\MakeUppercase{\@authorlast} \@authorfirst \\
		\vspace{1em}
		\textbf{\MakeUppercase{\@title}} \\
		\vspace{2em}
		Магистерская диссертация \\
		специальнось \@specialty
	\end{center}
																																																														
	\vspace{2em}
	\begin{flushright}
		\begin{minipage}[H]{0.4\textwidth}
			\begin{flushleft}
				Научный руководитель \\
				\@mentor \\
				\@mentorjob
			\end{flushleft}
		\end{minipage}
	\end{flushright}
																																																			
	\vspace{3em}
																																																			
	\vfill
																																																					
	\begin{flushleft}
		\begin{minipage}[H]{0.5\textwidth}
			\begin{flushleft}
				Допущено к защите \\
				<<\rule{1cm}{1pt}>> \rule{4cm}{1pt} \the\year г. \\
				Зав. Кафедрой \\
				\rule{6cm}{1pt} \\
				\rule{6cm}{1pt}
			\end{flushleft}
		\end{minipage}
	\end{flushleft}
																																																														
	\vspace{1em}
																																																														  
	\begin{center}
		Минск, \the\year
	\end{center}
\end{titlepage}


% ABSTRACT

% \begin{abstract}
% 	Аннотация на русском.
% \end{abstract}

% \selectlanguage{english}
    
% \begin{abstract}
% 	Abstract in English.
% \end{abstract}
    
\selectlanguage{russian}

% Общая характеристика работы
% «Общая характеристика работы» содержит:
% перечень ключевых слов; 
% цель, задачи, объект и предмет исследования;
% формулиро��ку полученных ре��ультато�� и и�� новизну;
% сведения о стр��кту��е магистерской диссертации. 
% Перечень ключевых слов характеризует основное содержание магистерской диссертации и включает 10-15 слов в именител/ьном падеже, написанных через запятую в строку прописными буквами. 

\fakechapter{Реферат}

Работа: N стр., N табл., N рис., N источников, N приложений

\MakeUppercase{временные ряды, модели с переключением состояний, анализ поворотных точек экономических циклов}

Объект исследования -- математические модели временных рядов, включающие переключений состояния.

TODO: Цель работы -- исследование ...

Методы исследования -- компьютерное моделирование временных рядов как сгенерированных, так и индикаторов экономики Республики Беларусь.

TODO: В результате исследования...

% ОГЛАВЛЕНИЕ

\clearpage
% Название содержания меняем
\renewcommand{\contentsname}{Содержание}
\tableofcontents

% Перечень условных обозначений, символов и терминов
\fakechapter{Условные обозначения и термины}

\section*{Термины и сокращения}

\textbf{AR, VAR, VARX} -- авторегрессионные модели, соответственно: одномерная, векторная, векторная с экзогенными переменными.

\textbf{IS-} -- модель с независимым переключением состояний (от англ. independent switching).

\textbf{MS-} -- модель с Марковским переключением состояний (от англ. Markov switching).

\textbf{GDP, ESI} -- макроэкономические индикаторы, ВВП и ИЭН (индекс экономических настроений) соответственно.

\section*{Обозначения в формулах}

$y$ -- эндогенная (моделируемая) переменная, случайная величина, возможно векторная.

$x$ -- экзогенная переменная, возможно векторная.

$y_t, \: x_t$ -- реализация переменных в момент времени $t$.

$\hat{y}_t$ -- прогноз эндогенной переменной на момент времени $t$.

$l_t = l(t) \in \overline{1,L}$ -- латентная (ненаблюдаемая) переменная состояния.

$L$ -- количество классов состояния (режимов).


% Введение
% Во «Введении» (объем до 3 страниц) обосновывается актуальность темы, ее значение, выбор направления исследования, показывается необходимость проведения исследований по данной теме для решения конкретной проблемы (задачи), развития конкретных направлений в соответствующих областях науки, отраслях экономики
\fakechapter{Введение}

В работе представляется многосторннее исследование моделей временных рядов с переключением состояний (regime-switching models), в частности авторегрессионные модели (семейства VARX) с независимым (IS-) или Марковским (MS-) переключением состояний.

\dots


% Основная часть
\chapter{Обзорная часть}
% TODO: Rename

\section{Модели с переключением}

Модели со скрытыми переменными (latent variables) по определению включают в себя структурное предложение: в моделируемом процессе существует хотя бы одна переменная, которая не наблюдается или не может наблюдаться, но которая влияет на наблюдаемые/измеримые переменные. Такие модели используются в экономике, машинном обучении (особенно при обработке естественного языка), биологии, психологии и т.д.

В данной работе рассматриваются модели временных рядов, где скрытая переменная является дискретной и категориальной с возможными значениями $l \in \overline{1,L} = \{1,2,\dots,L\}$. Интерпретация этой переменной -- класс состояния (regime), который определяет подмодель, по которой определяются другие переменные. Чаще всего эти подмодели имеют одинаковую структуру, но параметры оцениваются отдельно для каждого класса. В таких случаях модели с переключением состояний (regime switching / $RS$ models) – подкласс моделей со структурными изменениями.

Одно важное место применения RS-модели -- оценка и прогнозирование экономических циклов. Каждое отдельное состояние $l$ интерпретируется как часть экономического цикла. Например, при $L=2$, эти два класса могут интерпретироваться в качестве фаз роста и спада.


\section{Предыдущая работа автора}




\chapter{Эконометрическая теория RS-моделей}

\section{Общая характеристика}

Класс состояния в момент времени $t$ обозначается $l \in \overline{1,L}$. 


Самая общая формулировка моделей RS-моделей для временнях рядов следующая:

\[
	y_t \sim F_{l(t)}(t, y_{t-1}, \dots, y_0, x_t, x_{t-1}, \dots, x_0) 
\]
\[
	l(t) = l_t \sim S(t, l_{t-1}, \dots, l_0, y_{t-1}, \dots, y_0, x_t, \dots, x_0) 
\]

где $l(t) = l_t$ -- класс состояния в момент времени $t$, $S(\cdot)$ -- функция изменения состояния, а $\{F_l(\cdot)\}$ -- функции условново распределения $y$ при значении $l \in \overline{1,L}$. 

RS-модели, таким образом, включают очень широкий класс возможных моделей.
Однако на практике используются конкретные классы моделей. 

\section{Описание RS-VARX, IS-VARX, MS-VARS}

Особое внимание отделим авторегрессионным моделям RS-VARX, которые описываются следующей функцией распределения $y$:

\[
	y_{t}=\alpha_{l(t)} + \sum_{i=1}^{p} A_{i,l(t)} y_{t-i} + B_{l(t)} x_{t} + \eta_{t, l(t)} 
\]

где 
$y_t$ -- эндогенный вектор размерности $n$, 
$x_{t}$ -- экзогенная векторная переменная размерности $m$,
$p$ -- порядок авторегрессии, 
$\alpha_{l}$ -- вектор констант,
$A_{i,l}$ ($n \times n$ матрицы) и $B_{l}$ ($n \times m$ матрицы) -- матрицы авторегрессии и регрессии соответственно,
и $\eta_{t, l} \sim N_n(0, \Sigma_{l}) $ -- нормально–распределенный белый шум.
Все переменные, индексированные с $l$, могут зависеть от текущего состояния (<<переключаются>>); для уменьшения количества параметров, возможно убрать это переключение для некоторых из переменных.

На процесс состояния и $S(\cdot)$ также добавляются предположения. Широко используются <<Independent switching>> и <<Markov switching>> модели:

$ y_t \sim IS $

$ y_t \sim MS $


\section{Экономические цикли}



\chapter{Экспериментальные исследования на данных экономики РБ}

\dots

\chapter{Применение методов машинного обучения к чему-то}

\dots

\chapter{Программная реализация}

% TODO: 
\dots

% Заключение (выводы)
\fakechapter{Заключение}

% TODO: Conclusion.
\dots

% Список использованных источников
\fakechapter{Список использованных источников}
\printbibliography[title=Список использованных ����точников]

% Граф��ческий материал

% Программы (при необходимости)

% Приложения (при необходимости)


\end{document}